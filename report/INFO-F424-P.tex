\documentclass[a4paper,10pt]{article}
\usepackage[utf8]{inputenc}
\usepackage[a4paper,left=2.7cm,right=2.7cm,top=2.7cm,bottom=2.7cm]{geometry}
\usepackage{parskip}
\usepackage{eurosym}
\usepackage{amsmath}
\usepackage{graphicx}
\usepackage{hyperref}
\usepackage{tikz}
\usepackage{listings}
\usetikzlibrary{positioning}

\usepackage{csvsimple}

% 1
\begin{filecontents*}{1_1_alg.csv}
initialsolution,neighbourhood,pivotingrule,ARCD,ACT
Random,Exchange,Best,36.50,23.43
Random,Exchange,First,42.66,9.70
\end{filecontents*}


%%%
%%%
%%%
\title{INFO-F424 - Combinatorial Optimization\\Project - The $p$-Center Problem}
\date{\vspace{-3ex}
Erica \textsc{Berghman} \\
Charles \textsc{Hamesse} \\
~\\
École Polytechnique de Bruxelles\\
\vspace{6ex}May 2017\vspace{4ex}}
\begin{document}


% You must implement these formulations in Julia language combined with JuMP package.
% You must prepare a project report written in LATEX. In this report you should describe the mathematical formulations referenced above (explaining the meaning of each variable and constraint set), describe the computational experiments, and discuss the results. Performance of the implementation is also taken into consideration in the grading.

\maketitle
\begin{abstract}
    The purpose of this project is to implement two formulations of the same combinatorial optimization problem in Julia, using the JuMP package. We will start by describing the mathematical aspects of both formulations, then we will explain our implementations and discuss their performance.\vspace{2ex}
\end{abstract}

\tableofcontents
\pagebreak

% You must send the report and code to guillerme.duvillie@ulb.ac.be and leave a physical copy at the Secr ́etariat des E ́tudiants du D ́epartment d’Informatique at the 8th floor of the NO building, by 8th of May.

% Ease of use (read/write on the standard input, options, CLI, etc) is taken into consideration in the grading 

%%%
%%%	
%%%
\section{Introduction}

\subsection{Implementation}

\subsection{Compiling and running}
 
\paragraph{Compiling}

\paragraph{Running}
Some example runs:
\begin{lstlisting}
code
\end{lstlisting}

%%%
%%%
%%%
\section{Daskin (1995)}
	This is the formulation referred to as (P1) in the original paper.
	
	\subsection{Description}
	According to the usual canvas, the mathematical formulation is given as follows.
	\paragraph{Variables} Three variables are used in this formulation:
	\begin{eqnarray*}
		y_j &=& \begin{cases}
 				1 ~~\text{if vertex $j$ is a center} \\
 				0 ~~\text{otherwise}
 			\end{cases} \\
 		x_{ij} &=& \begin{cases}
 				1 ~~\text{if vertex $i$ assigns to a center in vertex $j$} \\
 				0 ~~\text{otherwise}
 			\end{cases} \\
 		p &=& \text{maximum number of centers}
	\end{eqnarray*}
	Both indices $i$ and $j$ have a range of $[1, N]$ where $N$ is the number of vertices of the instance. 
	
	\paragraph{Objective function}
	\begin{eqnarray}
		min && z\\
		\text{s.t.}~~~ \sum_{j \in N} d_{ij} x_{ij} &\leq& z 
	\end{eqnarray}
	These two expressions ensure that the objective value is no less than the maximum vertex-to-center distance, which we want to minimize. Note that (2) is actually implemented as a constraint but shown here for the sake of readability.
	
	\paragraph{Constraints}
	\begin{eqnarray}
		\sum_{j \in N} x_{ij} &=& 1 ~~\forall i \in N \\
		x_{ij} &\leq& y_i ~~\forall i,j \in N \\
		\sum_{j \in N} y_j &\leq& p \\
		y_j &\in& \{ 0,1 \} ~~\forall j \in N \\
		x_{ij} &\in& \{0 , 1 \} ~~\forall i,j \in N 
	\end{eqnarray}
	
	Constraint (3) assigns each vertex to exactly one center.
	Constraint (4) ensures that no vertex assigns to $v_j$ unless there is a center at $v_j$. 
	Constraint (5) restricts the number of centers to $p$.
	Constraints (6) and (7) are the binary restrictions for variables $x$ and $y$. 
	\subsection{Implementation}
		
	\subsection{Results}
	Example table:
    \begin{center}
    \begin{tabular}{l|l|l|r|r}%
    \bfseries Initial solution & \bfseries Neighbourhood & 
    \bfseries Pivoting rule & \bfseries ARCD & \bfseries ACT \\ \hline% specify table head
    \csvreader[head to column names]{1_1_alg.csv}{}% use head of csv as column names
    {\initialsolution & \neighbourhood & \pivotingrule & \ARCD & \ACT \\}% specify your coloumns here
    \end{tabular}
    \end{center}
    
    
\section{Calik and Tansel (2013)}
	This is the formulation referred to as (P3) in the original paper.
	\subsection{Description}
		
	\subsection{Implementation}
		
	\subsection{Results}
    
%%%
%%%
%%%

\section{Conclusion}
	\subsection{Comparison of the two formulations}
	\subsection{Wrap-up}
	

\end{document}




